\chapter{Background}\label{chapter:background}
% TODO : add introduction to background

\section{Linguistics}
Linguistics, the scientific study of languages is a broad and complex field encompassing various subfields. Although a comprehensive summary 
of Linguistics is beyond the scope of this thesis, we will briefly discuss some of the subfields and key concepts that are relevant to our work.

\subsection{Phonetics and Phonology}
\textbf{Phonetics} is the study of the physical sounds of human speech, their production, transmission and reception. \cite{trask2007language}. The International Phonetic Alphabet (IPA) is a standardized system of phonetic notation that represents the sounds of spoken language. The system
is based on the assumption that speech can be represented partly as a sequence of discrete sounds or \textit{segments} \cite{handbookIPA1999}. 
In addition, the IPA also includes symbols for suprasegmental features such as stress and intonation. The full IPA Chart (reproduced here in 
\ref{fig:ipa_chart}) shows all the symbols and diacritics used to represent sounds in the IPA. Sounds and words can be transcribed in IPA 
using \textipa{[ ]} brackets. For example, the sounds for the word \textit{this} can be transcribed as \textipa{[DIs]}. The IPA helps linguistics
transcribe sounds in a language-agnostic way, allowing them to compare sounds across languages.

\textbf{Phonology} is the study of the sound systems of languages, including the patterns of sounds and the rules that govern their distribution. \cite{trask2007language}.
The key difference in the disciplines is driven by the concept of a \textit{phoneme}. A phoneme is an abstract unit of sound that can distinguished
by a native speaker of a language. Phonemes and Phonemic transcriptions are represented using slashes \textipa{/ /}. The key points about phonemes are:
\begin{enumerate}
    \item Letters do not necessarily correspond to phonemes. For example, the English word \textit{this} has four letters but 3 phonemes (\textipa{/DIs/}).
    \item Phonemes can be realized as different sounds in different contexts. For example, the English phoneme \textipa{/p/} can be realized as
    \textipa{[p\super{h}]} in the word \textit{pin}(\textipa{[p\super{h}In]}) and \textipa{[p]} in the word \textit{spin}(\textipa{[spIn]}).
    i.e. in English, the sounds \textipa{[p]} and \textipa{[p\super{h}]} are \textit{allophones} of the phoneme \textipa{/p/}.
    \item Two sounds are considered different phonemes if changing them can change the meaning of a word. e.g. \textipa{[dEn]} \textit{den} and 
    \textipa{[DEn]} \textit{then} are distinct words in English.
\end{enumerate}

% From Wikipedia : In phonetics, the smallest perceptible segment is a phone. In phonology, there is a subfield of segmental phonology that deals with the 
% analysis of speech into phonemes (or segmental phonemes), which correspond fairly well to phonetic segments of the analysed speech.
% Also diff between segments and phonemes and phones?
% Should datasets like Phoible be mentioned here?

\textbf{Phonotactics} defines the rules that govern the permissible sound sequences in a language \cite{trask2007language}. For example, in English,
the sequence \textipa{/bl/} is permissible at the beginning of a word (e.g. \textit{bled}) but not the sequence \textipa{/bn/}. Languages usually
modify loadwords to fit their own phonotactic constraints. For example, the English word \textit{beer} is borrowed into Japanese as \textit{biru}.


\subsection{Grammar}

\subsubsection{Morphology}
\textbf{Morphology} is the study of the structure of words and the rules that govern the formation of words in a language \cite{trask2007language}.
Most studies of morphology focus on the concept of a \textit{morpheme}, the smallest unit of meaning in a language. For example, the word \textit{unhappiness}
can be broken down into three morphemes: \textit{un-}, \textit{happy} and \textit{-ness}. Morphemes can be free or bound. Free morphemes can stand
alone as words (e.g. \textit{happy}) while bound morphemes must be attached to other morphemes (e.g. \textit{-ness}).

Morphology can be further divided into \textbf{inflectional} and \textbf{derivational} morphology. \textbf{Inflectional morphology} involves the
modification of a word for grammatical purposes such as tense, aspect, mood, number, e.g. the English verb \textit{walk} can be inflected to
\textit{walked}, \textit{walks}, \textit{walking}, etc. \textbf{Derivational morphology} involves the creation of new words from existing words.
For example, the English noun \textit{happiness} can be derived from the adjective \textit{happiness} by adding the suffix \textit{-ness}.

\subsubsection{Syntax}
\textbf{Syntax} is the study of the structure of sentences and the rules that govern the formation of sentences in a language \cite{trask2007language}.

The words that are used in a language, are divided into several \textit{word classes}, or \textit{parts of speech}. The most common parts of speech
are \textit{nouns}, \textit{verbs}, \textit{adjectives}, and \textit{adverbs}. Sometimes, parts of speech exhibit multiple forms, used in different
grammatical circumstances. Each of these forms indicate a certain \textit{grammatical category} or \textit{feature}. For example, in English, verbs
can be inflected for tense, aspect, mood, person, number, etc.

A language may choose to explicitly mark these features, i.e. \textit{grammaticalize} them \cite{rosenfelder2010language}. All features can be
expressed in any language (perhaps by adding explicit information), but every language chooses to express only a subset of these features grammatically.


\section{Constructed Languages}

\section{Interpretability of Large Language Models}

