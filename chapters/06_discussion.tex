\chapter{Discussion, Conclusion and Future Work}\label{chapter:discussion}

In this chapter, we discuss the results of the experiments conducted in Chapter \ref{chapter:results}. We will discuss the effect of different parameters on the performance of the models, 
and how they relate to the original goals of this thesis. We then conclude with a summary of the findings and suggest future work that can be done.

\section{Effect of Phoneme Count}

\begin{figure}[H]  
    \centering
    \includegraphics[width=0.7\linewidth]{figures/results/1_effect_of_phoneme_count.png}
    \caption{Comparing the Effects of different phoneme counts}
    \label{fig:compare-phoneme-count}
\end{figure}

As we can see in Figure \ref{fig:compare-phoneme-count}, the phoneme count does not seem to have a significant effect on the translation scores
or the Race-C scores. This being the case, we can conclude that we can simplify a language by reducing the phoneme count without affecting the 
ability of the language to convey meaning. 

\section{Effect of Phonotactics}

\begin{figure}[H]  
    \centering
    \includegraphics[width=0.7\linewidth]{figures/results/1_effect_of_phonotactics.png}
    \caption{Comparing the Effects of phonotactics}
    \label{fig:compare-phonotactics}
\end{figure}

Figure \ref{fig:compare-phonotactics} shows the effect of phonotactics on the translation scores and the Race-C scores. Again, we can see that
the phonotactics do not seem to have a significant effect either of the metrics. We can therefore conclude that we can simplify a language by 
using simplifed phonotactic rules without affecting the ability of the language to convey meaning.

\section{Effect of Grammar Rules}

\begin{figure}[H]  
    \centering
    \includegraphics[width=0.7\linewidth]{figures/results/1_effect_of_grammar.png}
    \caption{Comparing different Grammatical Rules}
    \label{fig:compare-grammar}
\end{figure}



\section{Effect of Vocabulary Generation method}

Figure \ref{fig:compare-vocab-gen-types} shows the effect of different vocabulary generation methods against our metrics.

\section{Effect of Language Model Temperature}



\section{Conclusion}


\section{Future Work}
The results of this thesis show that it is possible to generate simplified languages that are still able to convey meaning. However, there are still many
open areas for future work. Firstly, we can explore other different language generation methods, including more or less complex grammatical rules,
or different vocabulary generation methods. In addition, we can explore more metrics to evaluate the generated languages. In particular, there is more to explore
in terms of simplicity measures, which are complex to define and measure.
